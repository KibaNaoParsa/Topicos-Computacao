\documentclass[12pt]{article}

\usepackage[utf8]{inputenc}
\usepackage[english,portuguese]{babel}
\usepackage{color}
\usepackage{amsmath}
\usepackage{graphicx}

\title{A arte da Gambiarra}
\author{
	Elyas Correa Nogueira\\
	Departamento de Computação e Engenharia Civil\\
	CEFET-MG Unidade \underline{Varginha}
	\and
	Noobilson El Noobon\\
	Departamento de \texttt{Gambiarras}\\
	Escola \textit{Imaginária} do Nada com Nada, \textbf{Varginha}
}
\date{\today}



\begin{document}
	
	\maketitle

	\selectlanguage{english}
	
	\begin{abstract}
		{\color{red} Somebody once told me the world is gonna roll me,}
		
		I ain't the sharpest tool in the shed.
		
		She was looking kind of dumb, with her finger and her thumb,
		
		and the shape of an L on her forehead.
	\end{abstract}

	\selectlanguage{portuguese}

	\section{Introdução}
	
		Graças ao pacote \texttt{color}, posso usar cores como o {\color{blue}azul}
		
		Referenciamos na figura~\ref{mole} um cachorro da raça eh mole kkkkkkkkkkk
		
		\begin{figure}[h]
			\begin{center}
				\includegraphics[width=0.75\textwidth]{ehmole.jpg}
				\caption{eh mole kkkkkkkkkkkkkkkkkk}
				\label{mole}
			\end{center}
		\end{figure}
		
		Apresentamos Trabalhos Anteriores na Seção~\ref{trabalhos_anteriores}.


	\section{Trabalhos Anteriores}
		\label{trabalhos_anteriores}

		Nessa seção vamos abordar trabalhos anteriores.
		
		Einstein apresentou a Equação~\eqref{equacao_einstein}, onde $E$ representa energia, $m$ representa massa e $c$ a constante que representa um limite intransponível --- a velocidade da luz.
	
	\begin{equation}
		\label{equacao_einstein}
		E = mc^2		
	\end{equation}

	
\end{document}


% Organização de imagens
% h = here
% t = top
% p = page
% b = bottom
% ! = relax