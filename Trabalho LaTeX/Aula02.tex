\title{Nossa Terceira aula de TEC}
\author{Elyas Correa Nogueira}
\date{\today}


\documentclass[12pt]{article}

\usepackage[utf8]{inputenc} 
\usepackage{color}


\begin{document}
	
	\maketitle
	
	%\chapter{Capítulo}
	
	\section{Introdução}
	
	Nessa seção, vamos demonstrar o uso de subseções.
	
	Quero escrever uma fórmula matemática: $a^2 = b^2 + c^2$
	
	Outra: $f(x) = \frac{2 \times x}{4}$.
	
	\[ \lim x = \theta + \pi \]
	
	\[ \sum_{x=1}^{50} (x^2 + 1)\]
	Aqui vou continuar o texto.
	
	\subsection{Subseção Introdução - Primeira}
	
	Essa é a minha primeira {\color{red}subseção da introdução.} \\ Blablabla
	
	\subsubsection{Subsubseção da Primeira Seção da Introdução}
	
	Mais qualquer coisa aqui.
	
	\subsection{Subseção Introdução - Segunda}
	
	Mais um texto qualquer aqui.
	
	\section{Referencial Teórico}
	
	Segundo Gandhi, tudo depende do propósito.
	
\newpage	
	
	\appendix
	
	\section{Apêndice A}
	
	Aqui vamos provar que P != NP.
	
	\section{Apêndice B}
	
	Aqui vamos provar que todo número primo maior que 3 pode ser escrito como a soma de outros dois números primos.
	
	
\end{document}