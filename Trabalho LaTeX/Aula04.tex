\documentclass[12pt]{article}

\usepackage[utf8]{inputenc}
\usepackage[english,portuguese]{babel}
\usepackage{booktabs}
\usepackage{makecell}

\begin{document}
	
	Dados da população estão na tabela~\ref{tabela_pop}. Vou citar isso aqui:
	\cite{de2012harmonia}.
	
	Vamos falar de cidades\footnote{Somente cidades legais.}.
	
	\begin{table}[htb!]
		\centering
		\caption{Variação de população em cidades do Sul de Minas}
		\label{tabela_pop}
		\begin{tabular}{| l | c | c |}
			\toprule
			Cidade & População 2000 & População 2010 \\
				\midrule
				Varginha & 123.000 & 127.000 \\
				\midrule
				Paraguaçu & 20.000 & 22.000 \\ 
			\bottomrule	
		\end{tabular}
	
		\centering
		\caption{Variação de população em cidades do Sul de Minas}
		\label{tabela_pop2}
		\begin{tabular}{@{}lcc@{}}
			\toprule
			\thead{Cidade} & \thead{População 2000} & \thead{População 2010} \\
			\midrule
			\makecell{Wagner Aristides Machado da Silva \\ Pereira Júnior} & 123.000 & 127.000 \\
			Paraguaçu & 20.000 & 22.000 \\ 
			\bottomrule
		\end{tabular}
	\end{table}

	\bibliographystyle{abbrv}
	\bibliography{ref}
\end{document}